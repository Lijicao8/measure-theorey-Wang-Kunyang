% ========= common/style.tex =========
% 页面样式、目录深度、超链接、代码与算法样式等

% 页边距(如已在其他处设置 geometry,可删去)
\IfFileExists{geometry.sty}{\RequirePackage[a4paper,margin=1in]{geometry}}{}

% 超链接样式(配合 preamble 中 \usepackage[hidelinks]{hyperref})
% 如需彩色链接可改为 colorlinks 方案:
% \hypersetup{colorlinks=true, linkcolor=blue, citecolor=blue, urlcolor=blue}
\hypersetup{pdfauthor={},
            pdftitle={},
            pdfsubject={},
            pdfcreator={}, pdfproducer={},
            pdfkeywords={}
}

% 目录与编号深度
\setcounter{tocdepth}{2}   % 目录显示到 subsection
\setcounter{secnumdepth}{3}% 编号到 subsubsection

% 段落样式(按需)
% \setlength{\parindent}{2em}
% \setlength{\parskip}{0.25em}

% 列表间距更紧凑一些
\setlist[itemize]{topsep=2pt,itemsep=2pt,parsep=0pt}
\setlist[enumerate]{topsep=2pt,itemsep=2pt,parsep=0pt}

% 代码 listings 的统一风格
\lstset{
  basicstyle=\ttfamily\small,
  numbers=left,
  numberstyle=\tiny,
  stepnumber=1,
  numbersep=6pt,
  showstringspaces=false,
  breaklines=true,
  frame=single,
  tabsize=2,
  captionpos=b
}

% 算法环境(algorithmic 关键词本地化可按需更改)
% \renewcommand{\algorithmicrequire}{\textbf{输入:}}
% \renewcommand{\algorithmicensure}{\textbf{输出:}}

% TikZ 常用库(需要时解开)
% \usetikzlibrary{arrows.meta,calc,positioning,decorations.pathreplacing}

% cleveref(可选):更智能的引用,需要在 hyperref 之后加载
% \IfFileExists{cleveref.sty}{%
%   \RequirePackage[nameinlink]{cleveref}
%   % 中文标题映射(可按需配置)
%   \crefname{theorem}{定理}{定理}
%   \crefname{lemma}{引理}{引理}
%   \crefname{proposition}{命题}{命题}
%   \crefname{corollary}{推论}{推论}
%   \crefname{definition}{定义}{定义}
%   \crefname{example}{例}{例}
%   \crefname{equation}{式}{式}
% }{}

% =========================================================
%            中文正文用宋体;各级标题用黑体
% =========================================================

% 仅在 XeLaTeX/LuaLaTeX 下设置系统字体
\IfFileExists{fontspec.sty}{%
  % 正文字体:优先 SimSun(Windows),否则退到思源宋体
  \IfFontExistsTF{SimSun}{%
    \setCJKmainfont{SimSun}
  }{%
    \IfFontExistsTF{Noto Serif CJK SC}{%
      \setCJKmainfont{Noto Serif CJK SC}
    }{}
  }%
  % 黑体:优先 SimHei(Windows),否则退到思源黑体
  \IfFontExistsTF{SimHei}{%
    \setCJKfamilyfont{hei}{SimHei}
  }{%
    \IfFontExistsTF{Noto Sans CJK SC}{%
      \setCJKfamilyfont{hei}{Noto Sans CJK SC}
    }{}
  }%
}{}

% 确保正文默认用宋体
\AtBeginDocument{\songti}

% 使用 ctex 的标题样式接口,把所有层级标题设为“黑体 + 粗体”
\ctexset{
  chapter/format       = {\heiti\bfseries\zihao{-2}},  % 章标题
  section/format       = {\heiti\bfseries\Large},      % 节标题
  subsection/format    = {\heiti\bfseries\large},      % 小节
  subsubsection/format = {\heiti\bfseries\normalsize}, % 小小节
  paragraph/format     = {\heiti\bfseries\normalsize}, % 段标题
}

% (可选)目录中的标题也用黑体
\ctexset{
  contentsname = {目录},
  chapter/nameformat    = {\heiti\bfseries},
  section/nameformat    = {\heiti\bfseries},
  subsection/nameformat = {\heiti\bfseries},
}
