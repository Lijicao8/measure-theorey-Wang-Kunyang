\chapter{抽象测度和积分}
\section{测度}
\begin{definition}[$\sigma$环和$\sigma$代数]
    设$X$是一个集合,$\mathscr{R}$是$X$的一些子集构成的非空的族,若$\mathscr{R}$满足下述条件:

    (1)$A,B\in\mathscr{R}\Rightarrow A\backslash B\in\mathscr{R};$

    (2)$A_n\in\mathscr{R},n\in\mathbb{N}\Rightarrow \cup_{n=1}^\infty A_n\in\mathscr{R},$

    \noindent 则称$\mathscr{R}$是$X$上的$\sigma$环. 若$\mathscr{A}$是$X$上的$\sigma$环使得$X\in\mathscr{A}$,则称$\mathscr{A}$为$X$上的$\sigma$代数.
\end{definition}

注意,$\sigma$环$\mathscr{R}$一定包含$\emptyset$,由于
\begin{equation*}
    \bigcap_{n=1}^\infty A_n=\bigcup_{n=1}^\infty A_n\setminus\bigcup_{n=1}^\infty\left(\bigcup_{k=1}^\infty A_k\backslash A_n\right),
\end{equation*}
因此$\mathscr{R}$对于可列交运算也封闭.但是因为$X$可以不在$\mathscr{R}$中,因此$\mathscr{R}$可以不对余运算封闭.

\begin{definition}[可测空间]
    设$\mathscr{A}$是集$X$上的$\sigma$代数,把$X$与$\mathscr{A}$合起来叫做可测空间,记作$(X,\mathscr{A})$.
\end{definition}

\begin{definition}[测度、符号测度]
    设$\mathscr{R}$是$X$上的$\sigma$环,$\phi$是$\mathscr{R}$上的函数.如果$\phi$满足

    (1)$\phi(\emptyset)=0;$

    (2)$\phi$具有可列可加性,即当$A_n\in\mathscr{R},n\in\mathbb{N}$且$A_m\cap A_n=\varnothing~~(\forall m,n\in$ $\mathbf{N},m\neq n)$时 $\varphi(\bigcup_{n=1}^{\infty}A_n)=\sum_{n=1}^{\infty}\varphi\left(A_n\right)$成立.

    \noindent 则称$\phi$是$\mathscr{R}$上的符号测度. 不取负值的符号测度叫做测度.
\end{definition}


\begin{definition}[测度空间、完全的测度空间]
    若$(X,\mathscr{A})$是可测空间,$\mu$是$\mathscr{A}$上的测度,则称三元组$(X,\mathscr{A},\mu)$为测度空间,并称$\mathscr{A}$中的元为$\mu$可测集.

    一个测度空间$(X,\mathscr{A},\mu)$如果使零测度集每个子集都可测,就叫做完全的.
\end{definition}

\begin{definition}[有限测度、$\sigma$有限测度]
    设$(X,\mathscr{A})$是可测空间,$\phi$是$\mathscr{A}$上的符号测度.若$\phi$只取有限值,则称$\phi$为有限的.若$X$可表示为$\mathscr{A}$中可列个集$E_k$的并,$k\in\mathbb{N}$,使得$\phi(E_k)\in\mathbb{R},k\in\mathbb{N}$,则称$\phi$是$\sigma$有限的.
\end{definition}


\begin{proposition}[测度的一些基本性质]
设 $\mu$ 是 $\sigma$ 环 $\mathscr{R}$ 上的测度,则以下结论成立

    \begin{enumerate}
        \item  当 $A, B \in \mathscr{R}$ 且 $A \subset B$ 时,$\mu(A) \leqslant \mu(B)$.

        \item 若$\left\{A_n\right\}_{n=1}^\infty$是$\mathscr{R}$的 单 调 增 序 列 , 那 么
    $$\mu(\lim A_n)=\lim\mu(A_n).$$

    \item 若$\left\{\mathscr{A}_n\right\}_{n=1}^\infty$是$\mathscr{R}$中的单调降序列且$\mu(A_1)<\infty$,则

$$\mu(\lim A_n)=\lim\mu(A_n).$$

\item 若$A_n\in\mathscr{R},n\in\mathbf{N}$,那么
$$\mu(\underline{\lim}A_n)\leqslant\underline{\lim}\mu(A_n),$$
如果还知道 $\mu(\bigcup_{n=1}^{\infty}A_n)<\infty$,则
$$\mu(\overline{\lim}A_n)\geqslant\overline{\lim}\mu(A_n).$$
\end{enumerate}
\end{proposition}






\section{可测函数、积分}
\begin{definition}[可测函数]
    设$(X,\mathscr{A})$是可测空间,$E\in\mathscr{A}$. 设$f$是$E$上的广义实值函数. 如果对任意$a\in\mathbb{R}$,$\{x\in E: f(x)>a\}\in\mathscr{A}$,则称$f$是$E$上的$\mathscr{A}$可测函数.
\end{definition}

\begin{definition}[简单函数]
    设$(X,\mathscr{A})$是可测空间,$E_1,\dots,E_n$是$\mathscr{A}$中$n$个两两不交的集合,使得$\cup_{i=1}^nE_i=X$,称
    \begin{equation*}
        \phi(x)=\sum_{i=1}^n a_i\mathcal{X}_{E_i}(x),~~x\in X
    \end{equation*}
    为简单函数.
\end{definition}


\begin{theorem}[简单函数逼近非负可测函数]
    设$f$是可测集$E$上的非负可测函数,则存在一列非负简单函数 $f_n,n\in\mathbb{N}$,使得在$E$上$f_{n}\uparrow f$.
\end{theorem}

\begin{theorem}[Eropob]
    设$(X,\mathscr{A},\mu)$是测度空间,$E\in\mathscr{A}$且$\mu(E)<\infty$. 设$f_n$是$E$上的可测的$\mu-a.e.$有限函数.若$\{f_n\}$在$E$上$\mu-a.e.$收敛,那么,对任意$\epsilon>0$,存在$A\subset E,A\in\mathscr{A}$,使得$\mu(E\backslash A)<\epsilon$并且$\{f_n\}$在$A$上一致收敛.
\end{theorem}


\begin{definition}[依测度收敛]
    设$(X,\mathscr{A},\mu)$是测度空间,$f$和$f_n$是$X$上$\mathscr{A}$可测函数,若
    \begin{equation*}
        \forall\delta>0,\lim_{n\to\infty}\mu(\{x\in X{:}|f(x)-f_n(x)\mid\geqslant\delta\})=0,
    \end{equation*}
    则称$f_n$依测度$\mu$收敛到$f$,记作$f_n\xrightarrow{\mu}f.$
\end{definition}


\begin{theorem}[F.Riesz]
    若$f_n\xrightarrow{\mu}f$,则有子列$\{f_{n_k}\}_{k=1}^\infty$ ~$\mu-a.e.$收敛到$f$.
\end{theorem}



\begin{theorem}[Levi]\label{thm:1.2.7}
设 $k\in\mathbb{N}$,$f_k$ 可测且 $f_k\to f\ (k\to\infty)$.则
\begin{enumerate}
  \item 若 $\exists\,\varphi\in L(X,\mathscr{A},\mu)$ 使得 $f_k\uparrow f$ 且 $\varphi\le f_k$(对一切 $k$),则
  \[
    \int_X f_k\,\mathrm d\mu \uparrow \int_X f\,\mathrm d\mu .
  \]
  \item 若 $\exists\,\varphi\in L(X,\mathscr{A},\mu)$ 使得 $f_k\downarrow f$ 且 $\varphi\ge f_k$(对一切 $k$),则
  \[
    \int_X f_k\,\mathrm d\mu \downarrow \int_X f\,\mathrm d\mu .
  \]
\end{enumerate}
\end{theorem}

\begin{theorem}[Fatou]\label{thm:1.2.8}
设 $k\in\mathbb{N}$,$f_k$ 可测,则
\begin{enumerate}
  \item 若 $\exists\,\varphi\in L(X,\mathscr{A},\mu)$ 使得 $\varphi\ge f_k$(对一切 $k$),则
  \[
    \int_X \liminf_{k\to\infty} f_k\,\mathrm d\mu
    \le
    \liminf_{k\to\infty}\int_X f_k\,\mathrm d\mu .
  \]
  \item 若 $\exists\,\varphi\in L(X,\mathscr{A},\mu)$ 使得 $\varphi\le f_k$(对一切 $k$),则
  \[
    \int_X \limsup_{k\to\infty} f_k\,\mathrm d\mu
    \ge
    \limsup_{k\to\infty}\int_X f_k\,\mathrm d\mu .
  \]
\end{enumerate}
\end{theorem}

\begin{theorem}[Lebesgue 支配收敛]\label{thm:1.2.9}
设 $f_k$ 为 $\mathscr{A}$-可测并满足 $f_k\to f$.若存在 $\varphi\in L(X,\mathscr{A},\mu)$ 使得
$|f_k|\le \varphi$(对所有 $k$),则
\[
  \int_X f_k\,\mathrm d\mu \longrightarrow \int_X f\,\mathrm d\mu \qquad (k\to\infty).
\]
\end{theorem}

\begin{theorem}\label{thm:1.2.10}
设 $f\in L(X,\mathscr{A},\mu)$,则对任意 $\varepsilon>0$,存在 $\delta>0$,使得对一切 $A\in\mathscr{A}$,
只要 $\mu(A)<\delta$,就有
\[
  \int_X |f|\,\chi_A\,\mathrm d\mu < \varepsilon .
\]
\end{theorem}

\noindent\textbf{注}\; 以后我们把 $\displaystyle\int_X f\chi_A\,\mathrm d\mu$ 写作 $\displaystyle\int_A f\,\mathrm d\mu$.








