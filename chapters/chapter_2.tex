\chapter{测度与拓扑}

\section{拓扑空间与连续映射}

\subsection{拓扑空间}
\begin{definition}[拓扑]
    设 $X$ 是一个不空的集合.所谓 $X$ 上的一个拓扑,
是指 $X$ 的一个子集族 $\mathscr{T}$,它满足以下三个条件:

\begin{enumerate}
  \item[(i)] $\varnothing \in \mathscr{T},\ X\in \mathscr{T}$;
  \item[(ii)] 若 $\{U_{\alpha}\}\subset \mathscr{T}$,则 $\bigcup_{\alpha} U_{\alpha}\in \mathscr{T}$;
  \item[(iii)] 若 $\{U_i : i=1,\ldots,m\}\subset \mathscr{T}$,则
        $\displaystyle\bigcap_{i=1}^{m} U_i \in \mathscr{T}$.
\end{enumerate}
\end{definition}
\noindent 注意:拓扑不需要对余运算封闭.


\begin{definition}[开集、闭集、内部、闭包、紧性]
    对于拓扑空间$(X,\mathscr{T})$,$\mathscr{T}$中元素被称作开集,开集的余集叫做闭集.
    
    \noindent 设$A\subset X,$
    \begin{enumerate}
        \item 内部$$\mathring{A}=\bigcup_{V\subset A,V\in\mathscr{T}}V.$$
        \item 闭包$$\overline{A}=\bigcap_{F\supset A,F^\complement\in\mathscr{T}} F.$$
        \item 紧集:任意开覆盖有有限子覆盖.
    \end{enumerate}
\end{definition}

显然,紧集的闭子集也是紧集.但是紧集不一定是闭集,例如单点集永远是紧的,但可能不是闭集.例如对于集合$X=\{0,1\}$,以及拓扑$\mathscr{T}=\{\emptyset,\{1\},X\}$,单点集$\{1\}$就是开集.

\begin{definition}[子空间拓扑]
    设$\left\{X,\mathscr{T}\right\}$是拓扑空间,$Y$ 是$X$ 的 非 空 子 集 . 定 义 $\mathscr{T} ^\prime = \left \{ G\bigcap Y; \right .$ $G\in\mathscr{T}\}$,那么$,(Y,\mathscr{T}^\prime)$是拓扑空间,叫作$(X,\mathscr{T})$的子空间$,\mathscr{T}^\prime$叫作$\mathscr{T}$的子拓扑.
\end{definition}


\begin{definition}[Hausdorff空间与第二类分离性]
    设$X$是拓扑空间,若对于$X$中任意不同的两点$x,y$,必存在不相交开集$U\cap V=\emptyset$使得$x\in U,y\in V$,则称$X$为Hausdorff空间或$T_2$空间.定义中所述的性质叫做第二类分离性,常叫做$T_2$公理.
\end{definition}


\begin{theorem}
    在Hausdorff空间中,紧集一定是闭的.
\end{theorem}
\noindent 证明紧集的补集是开集即可,即对任意$y\in X\backslash K$,都存在开邻域$U_y\subset X\backslash K$.


关于Hausdorff空间有如下重要的性质:任意互不相交的紧集能被一对互不相交的开集分离.
\begin{theorem}
    设$X$为 Hausdorff 空间,$K$和$L$为$X$的紧集,且$K\bigcap L=\varnothing$, 那么,存在两开集$U$和 V,使得
    $$K\subset U,L\subset V,U\cap V=\emptyset.$$
\end{theorem}

\begin{corollary}
    设$X$为 Hausdorff 空间.若$K$为$X$的紧集,$U_1$和$U_2$为$X$的开集,且 $K\subset U_1\bigcup U_2$,则存在紧集 $K_1$ 和 $K_2$,使得 $K_1\subset U_1,K_2\subset U_2$,以及$K=K_1\bigcup K_2.$
\end{corollary}




\begin{definition}[正规拓扑空间]
    设$X$为拓扑空间,若$X$的任意一对互不相交的闭集$A\cap B = \emptyset$,存在互不相交的开集$U\cap V =\emptyset$,使得$A\subset U,B\subset B$,则称$X$为正规的拓扑空间.上述分离性叫做第四类分离性,常称为$T_4$公理.
\end{definition}

\begin{corollary}
    紧的Hausdorff空间是正规的.
\end{corollary}
\noindent 正规性表述的是一种很强的分离性,也就是任意闭集都可以被开集分离,这在一般的$T_2$空间中是做不到的. 而紧空间的闭子集是紧的,所以紧Hausdorff空间中的闭集是紧的,可以被开集分离.


\subsection{连续映射}

\begin{lemma}[Urysohn 引理]\label{thm:urysohn}
设 $X$ 为正规的拓扑空间,$E$ 和 $F$ 为 $X$ 的闭集,且 $E\cap F=\varnothing$,
则存在连续函数 $f:X\to[0,1]$,使得
\[
f(x)=
\begin{cases}
0, & x\in E,\\[0.3em]
1, & x\in F.
\end{cases}
\]
\end{lemma}



\begin{theorem}[Urysohn 延拓定理]\label{thm:urysohn-extension}
设 $X$ 是正规的拓扑空间,$F$ 是 $X$ 的闭子集,
$\varphi:F\to\mathbb{R}$ 是有界连续函数,那么存在有界连续函数
$f:X\to\mathbb{R}$ 满足:
\[
\forall x\in F,\ f(x)=\varphi(x);\qquad
\sup\{\,|f(x)|:x\in X\,\}=\sup\{\,|\varphi(x)|:x\in F\,\}.
\]
\end{theorem}





\section{局部紧的Hausdorff空间上的连续函数}
\begin{definition}[局部紧空间]
     拓扑空间 $X$ 称为\emph{局部紧}的,是指对任一
$x\in X$,有开集 $U\ni x$,且 $\overline{U}$ 为紧集.
\end{definition}

显然,$\mathbb{R}^n$ 是局部紧的 Hausdorff 空间(依 Euclid 拓扑).
此外,紧的 Hausdorff 空间也是局部紧的.为书写简便起见,我们将局部紧的
Hausdorff 空间记作 LCHS(Locally compact Hausdorff space).
LCHS 有如下的重要性质.


\begin{theorem}
设 $X$ 为 LCHS,紧集 $K\subset X$,开集 $U\subset X$,且 $K\subset U$,
那么存在一开集 $V\subset X$,使得 $\overline{V}$ 为紧集,且
\[
K\subset V\subset \overline{V}\subset U.
\]
\end{theorem}
\noindent 这里的重点是$\overline{V}$也在$U$中.因为$\overline{V}$作为LCHS的子空间,就是紧的Hausdorff空间,因此是正规的.因此,就可以存在连续函数的延拓定理.





\begin{theorem}
设 $X$ 为 LCHS,$K$ 为 $X$ 的紧集,$U$ 为 $X$ 的开集,且 $K\subset U$,
那么存在一 $f\in C_c(X)$,使得
\[
\chi_K(x)\le f(x)\le \chi_U(x), \qquad x\in X,
\]
以及
\[
\operatorname{supp} f \subset U,
\]
其中 $\chi_E$ 表示集合 $E$ 的特征函数.
\end{theorem}


\begin{theorem}
设 $X$ 为一 LCHS,$f\in C_c(X)$,以及
\[
\operatorname{supp} f \subset \bigcup_{i=1}^m U_i,
\]
其中每一 $U_i$ 为 $X$ 的开集,那么存在 $f_i\in C_c(X)$,
$1\le i\le m$,使得
\[
f = \sum_{i=1}^m f_i,
\qquad
\operatorname{supp} f_i \subset U_i \quad (1\le i\le m).
\]
此外,若 $f\ge 0$,则 $f_i\ge 0\ (1\le i\le m)$.
\end{theorem}

\begin{theorem}[Tietze 延拓定理]
设 $X$ 为 LCHS,且 $K$ 为 $X$ 的紧集.若 $f\in C(K)$,
则存在 $\varphi\in C_c(X)$,使得 $\varphi$ 在 $K$ 上的限制
\[
\varphi\big|_K = f.
\]
\end{theorem}




\begin{definition}[拓扑基]
    设 $(X,\mathscr{T})$ 为拓扑空间,$\mathscr{U}\subset\mathscr{T}$.如果任一 $A\in\mathscr{T}$ 皆可表示成 $\mathscr{U}$ 中元的并,则称 $\mathscr{U}$ 为 $(X,\mathscr{T})$ 的基.
此外,若基 $\mathscr{U}$ 是可数集,那么就称 $X$ 为具有可数基
$\mathscr{U}$ 的拓扑空间.
\end{definition}


\begin{theorem}
    设 $X$ 为 LCHS.若 $X$ 有可数基 $\mathscr{U}$,则 $X$ 的每一开集可表示成可数个紧集的并集.
\end{theorem}




\section{Radon 测度与Riesz 表示定理}
